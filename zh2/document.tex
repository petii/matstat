\documentclass[]{article}

\usepackage[utf8]{inputenc}
\usepackage{t1enc}
\usepackage{listings}
\usepackage[hungarian]{babel}
\usepackage{amsmath}

%opening
\title{2. zh gyakorl\'o p\'eld\'ak}
\author{Vida P\'eter}

\begin{document}

\maketitle

\begin{abstract}

\end{abstract}

\section{feladat}
200 elemű mintánk van Bin(20,3p-0.5) eloszlásból. Az átlaguk 8. Adjunk a) 95\%-os, b) 97,5\%-os megbízhatósági konfidencia intervallumot p értékére. (Feltehető, hogy az átlag eloszlása normális).

\section{feladat}
Értelmezze az alábbi számítógépes programot és eredményt.
\begin{lstlisting}[language=R]
talal=0; m=500
for (i in 1:m) {
	a <- 5;sig <- 2;n <- 20
	xdat=rgamma(n,1/a,2)
	error <- qt(0.975,n-1)*sd(xdat)/sqrt(n)
	left <- mean(xdat)-error
	right <- mean(xdat)+error
	if (left<a && a<right) 
		talal=talal+1
}
\end{lstlisting}
Az eredmény: talal=456

\section{feladat}
Adott egy 3 elemű minta $X_1, X_2, X_3$. 
Két hipotézisünk van:
\[ \begin{aligned}
H0: & P(X_i=2)=\frac{1}{3}, & P(X_i=3)=\frac{1}{3},& P(X_i=10)=\frac{1}{3} \\
H1: & P(X_i=2)=\frac{1}{5}, & P(X_i=3)=\frac{1}{5},& P(X_i=10)=\frac{3}{5} \\
\end{aligned} \]
Határozzuk meg annak a próbának a tulajdonságait, amely akkor utasítja el a
nullhipotézist, ha legalább 2-szer jön ki a 10-es érték. 

\section{feladat}
Milyen opciói vannak a t.test függvénynek?
\end{document}
