\documentclass[]{article}

\usepackage[utf8]{inputenc}
\usepackage{t1enc}
\usepackage{listings}
\usepackage[hungarian]{babel}
\usepackage{amsmath}

%opening
\title{2. zh gyakorl\'o p\'eld\'ak}
\author{Vida P\'eter}

\begin{document}

\maketitle

\begin{abstract}

\end{abstract}

\section{feladat}
200 elemű mintánk van Bin(20,3p-0.5) eloszlásból. Az átlaguk 8. Adjunk a) 95\%-os, b) 97,5\%-os megbízhatósági konfidencia intervallumot p értékére. (Feltehető, hogy az átlag eloszlása normális). \\
a) $1-\alpha = 95\% \Rightarrow\alpha = 0.05$ \\
$\overline{x} = 8$ \\
$\delta^2 = 20$ \\
$z_{1 - \frac{\alpha}{2}} = z_{0.975} \approx 0.78$\\
Konfidencia intervallum: 
\(
\left[  - z_{0.975} \cdot \frac{}{} \right]
\)




\section{feladat}
Értelmezze az alábbi számítógépes programot és eredményt.
\begin{lstlisting}[language=R]
talal=0; m=500
for (i in 1:m) {
	a <- 5;sig <- 2;n <- 20
	xdat=rgamma(n,1/a,2)
	error <- qt(0.975,n-1)*sd(xdat)/sqrt(n)
	left <- mean(xdat)-error
	right <- mean(xdat)+error
	if (left<a && a<right) 
		talal=talal+1
}
\end{lstlisting}
Az eredmény: talal=456

500-szor megism\'etelj\"uk azt a k\'is\'erletet, hogy gamma eloszl\'asb\'ol vesz\"unk egy 20 elem\H u mintat, \'es ebb\H ol a 95\%-os konfidenciaintervallumot adunk meg.
A tal\'al eredm\'eny az, hogy az 500 ism\'etl\'esb\H ol h\'anyszor szerepelt az a=5 \'ert\'ek ebben az intervallumban.
  
\section{feladat}
Adott egy 3 elemű minta $X_1, X_2, X_3$. 
Két hipotézisünk van:
\[ \begin{aligned}
H_0: & P(X_i=2)=\frac{1}{3}, & P(X_i=3)=\frac{1}{3},& P(X_i=10)=\frac{1}{3} \\
H_1: & P(X_i=2)=\frac{1}{5}, & P(X_i=3)=\frac{1}{5},& P(X_i=10)=\frac{3}{5} \\
\end{aligned} \]
Határozzuk meg annak a próbának a tulajdonságait, amely akkor utasítja el a nullhipotézist, ha legalább 2-szer jön ki a 10-es érték. 

Tulajdons\'agok:
\begin{itemize}
	\item Terjedelem: elsőfajú hiba valószínűségek felső
	határa. \\
	Els\H ofaj\'u hiba valószínűsége: $\alpha = P_\theta(\aleph_k) (\theta \in \Theta_0)$ \\
	$\aleph_k:\text{ kritikus tartom\'any, azaz amire elutas\'itjuk a nullhipot\'ezist} $\\ 
	$\aleph_k= \left\{ (10,10,10), (10,10,k), (10,k,10), (k,10,10) | k\in {2,3} \right\}$ \\
	$\alpha = \frac{7}{3^3} = 0.\dot 2 5 \dot 9  \le 0.26 $ ???
	
	\item Konzisztencia: az erőfv. 1-hez tart ($\forall \theta \in \Theta_1$) ????????????
\end{itemize}


\section{feladat}
Milyen opciói vannak a t.test függvénynek? \\
t.test param\'eterei:
\begin{itemize}
	\item x : nem\"ures sz\'amvektor (adatok)
	\item y : opcion\'alis nem\"ures sz\'amvektor (adatok)
	\item alternative: mi az alternat\'iv hipot\'ezis. Lehets\'eges \'ert\'ekek:
	\begin{itemize}
		\item t (two.sided)
		\item g (greater)
		\item l (less)
	\end{itemize}
	\item mu : a val\'odi \'atlag (vagy k\"ul\"onbs\'eg k\'etmint\'as pr\'oba eset\'en)
	\item paired : logikai, p\'aros t-pr\'oba vagy nem]
	\item var.equal : logikai, egyenl\H o-e a variancia
	\item conf.level : konfidencia intervallum
	\item formula : 'baloldal ~ jobboldal' alak\'u formula, ahol a baloldal sz\'am adatok, a jobboldal pedig faktor ami a kapcsol\'od\'o csoportokat adja
	\item data : opcion\'alis m\'atrix vagy data frame vagy hasonl\'o amiben a 'formula'-beli v\'altoz\'ok vannak
	\item subset : opcion\'alis vektor, a mint\'anak milyen r\'eszhalmaz\'at haszn\'aljuk
	\item na.action : mi t\"ort\'enjen ha NA-val tal\'alkozunk
\end{itemize}
\end{document}
